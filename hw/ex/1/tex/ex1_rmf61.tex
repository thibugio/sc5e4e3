\documentclass{article}

\usepackage[margin=1in]{geometry}
\usepackage{amsmath, listings, array, longtable}
\begin{document}
\lstset{language=C,
        escapeinside={@~}{~@},
        tabsize=2,
        %escapechar=@,
        basicstyle=\ttfamily\small
}

\newcommand\opspaceL{\ }
\newcommand\opspaceR{\ \ }
\newcommand\por{\opspaceL\sym{||}\opspaceR}
\newcommand\pand{\opspaceL\sym{&&}\opspaceR}
\newcommand\pnot{\opspaceL\sym{!}\opspaceR}
\newcommand\peq{\opspaceL\sym{=}\opspaceR}

\newcommand\sym[1]{{\ttfamily\small#1}}
\newcommand\sub[1]{\ensuremath{\text{}_{#1}}}
\newcommand\rbr{\sym{>}}
\newcommand\lbr{\sym{<}}

\newcommand\nterm[1]{\lbr\sym{#1}\rbr}
\newcommand\ntermn[2]{\lbr\sym{#1}$\text{}_{#2}$\rbr}

\newcommand\es{\ensuremath{\emptyset}}
\newcommand\union{\ensuremath{\cup}\ }
\newcommand\bnfor{\ \ensuremath{|}\ }
\newcommand\rarrow{\opspaceL\ensuremath{\rightarrow}\opspaceR}

\newcommand\mmap[1]{\ensuremath{M_{#1}}}
\newcommand\fmap[1]{{\it #1}}

\newcommand\na{{\it N/A}}
\newcommand\et{\opspaceL\ensuremath{\wedge}\opspaceR}

\newcolumntype{P}[1]{>{\small\bgroup\ttfamily\bgroup}p{#1\textwidth}<{\egroup\egroup}}
\newcolumntype{C}[1]{>{\small\bgroup\ttfamily\bgroup}#1<{\egroup\egroup}}

\noindent
Rebecca Frederick\\
EECS 345\\
Written Exercise 1\\
\today\\
\begin{enumerate}
\renewcommand{\arraystretch}{1.5}
\item \begin{tabular}[t]{rcl}
        % operator precedence (high to low): (), !, &&, ||, ?:, =
        % right-associative operators: !, ?:, =
        % left-associative operators: &&, ||
        \nterm{C} & \rarrow & \lstinline`<V> = <D>` \bnfor \nterm{V}\\
        \nterm{V} & \rarrow & \sym{x} \bnfor \sym{y} \bnfor \sym{z} \\
        \nterm{D} & \rarrow & \lstinline`<E> ? <D> : <D>` \bnfor \nterm{E} \\
        \nterm{E} & \rarrow & \lstinline`<E> || <F>` \bnfor \nterm{F} \\
        \nterm{F} & \rarrow & \lstinline`<F> && <G>` \bnfor \nterm{G} \\
        \nterm{G} & \rarrow & \lstinline`!<G>` \bnfor \nterm{H} \\
        \nterm{H} & \rarrow & \lstinline`(<H>)` \bnfor \nterm{I} \\
        \nterm{I} & \rarrow & \sym{true} \bnfor \sym{false}
      \end{tabular}
\item Static Semantic Attributes: \\
\renewcommand{\arraystretch}{1}
\begin{tabular}[t]{C{l}C{c}C{l}l}
    type & \peq & \{integer, double\} & (synthesized)\\
    typetable(<var>) & \peq & \{integer, double, error\} & (inherited)\\
    inittable(<var>) & \peq & \{true, false, error\} & (inherited)\\
    typebinding & \peq & (<var>, \{integer, double\}) & (synthesized)\\
    initialized & \peq & (<var>, \{true, false\}) & (synthesized)
\end{tabular}\\

Attribute Rules:\\
\begin{longtable}[t]{P{1}}
{\it \ntermn{start}{1} \rarrow \ntermn{stmt}{3} ; \ntermn{start}{3} }\\
\ntermn{start}{1}.type := \na \\
\ntermn{start}{1}.typetable(<var>) := \ntermn{stmt}{3}.typetable\\
\ntermn{start}{1}.inittable(<var>) := \ntermn{stmt}{3}.initvar\\
\ntermn{start}{1}.typebinding := \na\\
\ntermn{start}{1}.initialized := \na\\\\

\ntermn{stmt}{3}.type := \na\\
\ntermn{stmt}{3}.typetable := \ntermn{start}{1}.typetable\\
\ntermn{stmt}{3}.inittable := \ntermn{start}{1}.inittable\\
\ntermn{stmt}{3}.typebinding := \na\\
\ntermn{stmt}{3}.initialized := \na\\

\ntermn{start}{3}.type := \na\\
\ntermn{start}{3}.typetable := \ntermn{stmt}{3}.typetable \union \ntermn{start}{1}.typetable\\
\ntermn{start}{3}.inittable := \ntermn{stmt}{3}.inittable \union \ntermn{start}{1}.inittable\\
\ntermn{start}{3}.typebinding := \na\\
\ntermn{start}{3}.initialized := \na\\

{\it \ntermn{start}{2} \rarrow \ntermn{stmt}{4}}\\
\ntermn{start}{2}.type := \na\\
\ntermn{start}{2}.typetable(<var>) := \es\\
\ntermn{start}{2}.inittable(<var>) := \es\\
\ntermn{start}{2}.typebinding := \na\\
\ntermn{start}{2}.initialized := \na\\\\

\ntermn{stmt}{4}.type := \na\\
\ntermn{stmt}{4}.typetable := \ntermn{start}{2}.typetable\\
\ntermn{stmt}{4}.inittable := \ntermn{start}{2}.inittable\\
\ntermn{stmt}{4}.typebinding := \na\\
\ntermn{stmt}{4}.initialized := \na\\

\ntermn{start}{3}.type := \na\\
{\it \ntermn{stmt}{1} \rarrow \ntermn{declare}{2}}\\
\ntermn{stmt}{1}.type := \na\\
\ntermn{stmt}{1}.typetable(<var>) := \es (inherited from \nterm{start})\\
\ntermn{stmt}{1}.inittable(<var>) := \es (inherited from \nterm{start})\\
\ntermn{stmt}{1}.typebinding := \na\\
\ntermn{stmt}{1}.initialized := \na\\\\

{\it \ntermn{stmt}{2} \rarrow \ntermn{assign}{2}}\\
\ntermn{stmt}{2}.type := \na\\
\ntermn{stmt}{2}.typetable(<var>) := \es (inherited from \nterm{start})\\
\ntermn{stmt}{2}.inittable(<var>) := \es (inherited from \nterm{start})\\
\ntermn{stmt}{2}.typebinding := \na\\
\ntermn{stmt}{2}.initialized := \na\\\\

\ntermn{declare}{1}.type := \\
\ntermn{declare}{1}.typetable(<var>) := \\
\ntermn{declare}{1}.inittable(<var>) := \\
\ntermn{declare}{1}.typebinding := \\
\ntermn{declare}{1}.initialized := \\\\

\ntermn{type}{1}.type := \\
\ntermn{type}{1}.typetable(<var>) := \\
\ntermn{type}{1}.inittable(<var>) := \\
\ntermn{type}{1}.typebinding := \\
\ntermn{type}{1}.initialized := \\\\

\ntermn{type}{2}.type := \\
\ntermn{type}{2}.typetable(<var>) := \\
\ntermn{type}{2}.inittable(<var>) := \\
\ntermn{type}{2}.typebinding := \\
\ntermn{type}{2}.initialized := \\\\

\ntermn{assign}{1}.type := \\
\ntermn{assign}{1}.typetable(<var>) := \\
\ntermn{assign}{1}.inittable(<var>) := \\
\ntermn{assign}{1}.typebinding := \\
\ntermn{assign}{1}.initialized := \\\\

\ntermn{expression}{1}.type := \\
\ntermn{expression}{1}.typetable(<var>) := \\
\ntermn{expression}{1}.inittable(<var>) := \\
\ntermn{expression}{1}.typebinding := \\
\ntermn{expression}{1}.initialized := \\\\

\ntermn{expression}{2}.type := \\
\ntermn{expression}{2}.typetable(<var>) := \\
\ntermn{expression}{2}.inittable(<var>) := \\
\ntermn{expression}{2}.typebinding := \\
\ntermn{expression}{2}.initialized := \\\\

\ntermn{expression}{3}.type := \\
\ntermn{expression}{3}.typetable(<var>) := \\
\ntermn{expression}{3}.inittable(<var>) := \\
\ntermn{expression}{3}.typebinding := \\
\ntermn{expression}{3}.initialized := \\\\

\ntermn{expression}{4}.type := \\
\ntermn{expression}{4}.typetable(<var>) := \\
\ntermn{expression}{4}.inittable(<var>) := \\
\ntermn{expression}{4}.typebinding := \\
\ntermn{expression}{4}.initialized := \\\\

\ntermn{expression}{5}.type := \\
\ntermn{expression}{5}.typetable(<var>) := \\
\ntermn{expression}{5}.inittable(<var>) := \\
\ntermn{expression}{5}.typebinding := \\
\ntermn{expression}{5}.initialized := \\\\

\ntermn{value}{1}.type := \\
\ntermn{value}{1}.typetable(<var>) := \\
\ntermn{value}{1}.inittable(<var>) := \\
\ntermn{value}{1}.typebinding := \\
\ntermn{value}{1}.initialized := \\\\

\ntermn{value}{2}.type := \\
\ntermn{value}{2}.typetable(<var>) := \\
\ntermn{value}{2}.inittable(<var>) := \\
\ntermn{value}{2}.typebinding := \\
\ntermn{value}{2}.initialized := \\\\

\ntermn{value}{3}.type := \\
\ntermn{value}{3}.typetable(<var>) := \\
\ntermn{value}{3}.inittable(<var>) := \\
\ntermn{value}{3}.typebinding := \\
\ntermn{value}{3}.initialized := \\

\ntermn{value}{4}.type := \\
\ntermn{value}{4}.typetable(<var>) := \\
\ntermn{value}{4}.inittable(<var>) := \\
\ntermn{value}{4}.typebinding := \\
\ntermn{value}{4}.initialized := \\
\caption{Attribute Rules}
\end{longtable}

\item \begin{enumerate}
    \renewcommand\theenumiii{\arabic{enumiii}}
    \item `The type of the expression must match the type of the variable in all assignment statements'\\ \begin{enumerate}
        \item \ntermn{assign}{1}: \nterm{var}.type = \ntermn{expression}{3}.type
        %\item \ntermn{expression}{1}: \ntermn{expression}{4}.type = \ntermn{expression}{5}.type
        \end{enumerate}
    \item `A variable must be declared before it is used'\\ \begin{enumerate}
        \item \ntermn{assign}{1}: \nterm{var}.typetable != `Error'
        \end{enumerate}
    \item `A variable must be assigned a value as its first use in the program'\\ \begin{enumerate}
        \item \ntermn{assign}{1}: if \nterm{var}.inittable = `Error' %expression->value->var
        \end{enumerate}
    \end{enumerate}
\item Loop Invariants: 
    \begin{description}
    \item [Outer (while) Loop Goal:] The elements $A[0\dots n-1]$ are sorted in non-decreasing order
    \item [Outer (while) Loop Invariant:] The elements $A[bound \dots n-1]$ are in non-decreasing order \et\\
                                          the elements $A[t\dots bound-1]$ have yet to be sorted.
                                          
                                          (The last condition may be redundant but I felt it necessary to include $t$ in the outer loop invariant since it
                                           is initiallized outside of the inner loop and also interacts with a variable ($bound$) in the outer loop.)
    \item [Inner (for) Loop Goal:] the elements $A[t\dots n-1]$ are sorted in non-decreasing order
    \item [Inner (for) Loop Invariant:] The elements $A[bound \dots n-1]$ are in non-decreasing order \et\\ 
                                        $A[0\dots i] \leq A[t] \et$ \\
                                        $A[t] \leq A[bound]$.
    \end{description}
{\bf Precondition:} $n \ge 0$ and $A$ contains $n$ elements indexed from 0
\begin{lstlisting}
bound = n; 
while (bound > 0) { 
  
  // Assume Outer Loop Invariant is true
  t = 0;
  
  for (i = 0; i < bound - 1; i++) { 
    
    // Assume Inner Loop Invariant is true
    if (A[i] > A[i+1]) {
      
      // WP (Inner):
      // @~$A[bound\dots n-1]$~@ are in non-decreasing order @~\et~@
      // @~$A[0\dots i-1] \leq A[i]$ \et~@
      // @~$A[i] \leq A[bound]$~@  
      swap = A[i]; 

      // WP (Inner):
      // @~$A[bound\dots n-1]$~@ are in non-decreasing order @~\et~@
      // @~$A[0\dots i-1] \leq A[i+1]$ \et~@
      // @~$A[i+1] \leq A[bound]$~@  
      A[i] = A[i+1];

      // WP (Inner):
      // @~$A[bound\dots n-1]$~@ are in non-decreasing order @~\et~@
      // @~$A[0\dots i] \leq$~@ swap @~\et~@
      // swap @~$\leq A[bound]$~@
      A[i+1] = swap;

      // WP (Inner):
      // @~$A[bound\dots n-1]$~@ are in non-decreasing order @~\et~@
      // @~$A[0\dots i] \leq A[i+1]$ \et~@
      // @~$A[i+1] \leq A[bound]$~@
      t = i + 1; 
    }
    // (loop termination: i=bound-1, t=`the last i+1 for which @~$A[i] > A[i+1]$~@')
    // i=bound-1 @~\et $A[t] \leq A[bound]$ \et $A[0\dots i] \leq A[t]$ \et~@
    //    @~$A[bound \dots n-1]$~@ are in non-decreasing order @~\rarrow~@
    //        @~$A[t\dots n-1]$~@ are sorted in non-decreasing order
    // i++
  }
  // WP (Outer): 
  // @~$A[bound\dots n-1]$~@ are in non-decreasing order @~\et~@
  //    @~$A[t\dots bound-1]$~@ have yet to be sorted
  bound = t;  
} 
// (loop termination: bound=0, t=0)
// bound=0 @~\et~@
// @~$A[0\dots n-1]$~@ are sorted in non-decreasing order @~\et~@
//    @~$A[0\dots -1]$~@ have yet to be sorted (trivially true) @~\rarrow~@ 
//        array @~$A$~@ is sorted in non-decreasing order
\end{lstlisting}
{\bf Postcondition:} $A[0] \le A[1] \le \dots \le A[n-1]$ (i.e., array $A$ is sorted in non-decreasing order)

\item \mmap{state}(\nterm{var} \peq \nterm{expression}, S) = \begin{lstlisting}
{
    // test that @~\nterm{var}~@ is a legal name in the language
    if @~\mmap{name}(\nterm{var}~@) = @~`Error'~@
        return @~`Error'~@
   
    // test that @~\nterm{var}~@ has already been declared
    if @~\fmap{Lookup}(\mmap{name}(\nterm{var}), $S$~@) = @~`Error'~@
        return @~`Error'~@

    // calculate the value of @~\nterm{expression}~@ using the old state
    @~$V$~@ = @~\mmap{value}(\nterm{expression}, $S$~@)
    if @~$V$~@ = @~`Error'~@
        return @~`Error'~@
   
    // calculate a new state including any side effects from evaluating @~\nterm{expression}~@
    @~$S_1$~@ = @~\mmap{state}(\nterm{expression}, $S$~@)

    // remove @~\nterm{var}~@ from the new state
    @~\fmap{Remove}(\mmap{name}(\nterm{var}), $S$~@)

    // return the new state with the updated value of @~\nterm{var}~@ added
    return @~\fmap{Add}(\mmap{name}(\nterm{var}), $V$, $S_1$~@)
             
}
\end{lstlisting}


\mmap{state}(if \nterm{condition} then \ntermn{statement}{1} else \ntermn{statement}{2}, S) = \begin{lstlisting}
{
    @~$S_1$~@ = @~\mmap{state}(\nterm{condition}, $S$)~@
    
    if @~\mmap{boolean}(\nterm{condition}, $S_1$~@) = true
        return @~\mmap{state}(\ntermn{statement}{1}, $S_1$~@)
    else if @~\mmap{boolean}(\nterm{condition}, $S_1$~@) = false 
        return @~\mmap{state}(\ntermn{statement}{2}, $S_1$~@)
    else
        return `Error'
}
\end{lstlisting}


\mmap{state}(while \nterm{condition} \nterm{loop body}, S) = \begin{lstlisting}
{
    @~$S_1$~@ = @~\mmap{state}(\nterm{condition}, $S$)~@
    
    if @~\mmap{boolean}(\nterm{condition}, $S_1$~@) = true
        // evaluate the loop body and call the while-loop again
        return @~\mmap{state}~@(while @~\nterm{condition} \nterm{loop body}, \mmap{state}(\nterm{loop body}, $S_1$~@))
    else if @~\mmap{boolean}(\nterm{condition}, $S_1$~@) = false 
        return @~$S_1$~@ 
    else
        return `Error'
}
\end{lstlisting}
\end{enumerate}
\end{document}
